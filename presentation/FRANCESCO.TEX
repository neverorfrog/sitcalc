%!TEX root = presentation.tex

% memo: mille modi per centrare
% https://tex.stackexchange.com/questions/449578/basic-how-can-i-stop-centering-a-text

\subsection{Introduction}

\begin{frame}{Outline}
    Situation calculus provides a framework for reasoning about actions.

    This work presents an expansion to handle the \textit{knowledge} possessed or acquired by the agent,
    and allow it to shape the agent's decisions.
    \begin{itemize}%[<+->]
        \item Knowledge is represented by one additional fluent
        \item Uniform axiomatization with the rest of sitcalc
        \item Ordinary actions and knowledge-producing ones are strictly separated
        \item Easy expansion of regression as defined in [Reiter2001]
        \item Desirable properties are direct consequences of the axiomatization \\
                (e.g. knowledge persistence / memory)
    \end{itemize}
\end{frame}

% opzionale
\begin{frame}{...}
    Opzionale

    Un paio di azioni ordinarie e un paio di azioni di conoscenza di esempio, giusto per inquadrare il discorso
\end{frame}

\subsection{Knowledge as a fluent}

\begin{frame}[fragile]{The K fluent}
    \huge
    \[ \sck(s', s) \]
    \normalsize

    Defines an accessibility relation between situations.

    % Informal definition: \( \sck(s', s) \) is true if, given its current knowledge,
    % the situations \(s\) and \(s'\) are indistinguishable to the agent.

    \begin{block}{(Informal) definition}
        \( \sck(s',s) \) is true if an agent in situation \(s\)
        could mistake the current situation for the other \(s'\),
        given its current knowledge.
    \end{block}
\end{frame}

\begin{frame}{Knowledge}
    \begin{block}{Definition of knowledge}
        A fluent is known to be true (false) in a situation \(s\)
        if and only if it is true (false)
        in all situations accessible from \(s\).
    \end{block}

    Shorthand notation: \( \knows(\phi, s) \defeq \forall s' \: \sck(s',s) \rightarrow \phi(s') \)
\end{frame}

\begin{frame}{Knowledge-producing actions}
    Actions that have an effect on the agent's knowledge

    \begin{block}{SENSE actions}
        % Learn whether a fluent is true or false. Example: check if a door is open or closed.
        Learn the truth value of a formula. Example: check if a door is open or closed.
        \[
            \knows(\textfluent{P}, \scdo(\scsense_\text{P}, s))
            \lor
            \knows(\lnot \textfluent{P}, \scdo(\scsense_\text{P}, s))
        \]
    \end{block}

    \begin{block}{READ actions}
        % Learn what a functional term refers to. Example: read a number on a sheet of paper.
        Learn the value of a term. Example: read a number on a sheet of paper.
        \[
            \exists x \: \knows(\tau = x, \scdo(\scread_\tau, s))
        \]
    \end{block}

    \emph{Assumption: ordinary and knowledge-producing actions are strictly separated.}
\end{frame}

\subsection{Defining a successor state axiom for K}

\begin{frame}{Knowledge effects}
    In order to complete the specification of the \sck fluent,
    we need to define its successor state axiom,
    determining how ordinary actions and knowledge-producing actions affect it.

    Consider this case study with three accessible situations. The agent is in S1.

    \begin{center}
        \includegraphics[width=0.6\textwidth]{assets/3states_noactions.png}
    \end{center}

    \[ \knows(\textfluent{P}, S1) \land \lnot \knows(\textfluent{Q}, S1) \]
\end{frame}

\begin{frame}{Ordinary actions}
    Assume the agent performs a \textfluent{drop} action.

    \begin{block}{Informal idea}
        The agent cannot distinguish the current situation \(s\) from all the other
        \(s'\) accessible from it. Therefore, after executing the action,
        the agent may believe to be in any situation resulting from any \(s'\) after executing \textfluent{drop}.
    \end{block}

    \begin{block}{Axiomatization}
        \[ \sck(s'', \scdo(\scdrop, s)) \equiv \exists s' \: (\scposs(\scdrop, s') \land \sck(s',s) \land s'' = \scdo(\scdrop, s')) \]
    \end{block}
\end{frame}

\begin{frame}{Ordinary actions}
    Pippo
\end{frame}

\begin{frame}{Knowledge-producing actions}
    Pippo
\end{frame}

\begin{frame}{The successor state axiom for K}
    Pippo
\end{frame}

\begin{frame}{<varie ed eventuali>}
    Pippo
\end{frame}
