%!TEX root = presentation.tex

\section{Introduction}

\begin{frame}{Outline}
    Situation calculus provides a framework for reasoning about actions.

    This work presents an expansion to handle the \textit{knowledge} possessed or acquired by the agent,
    and allow it to shape the agent's decisions.
    \begin{itemize}%[<+->]
        \item Knowledge is represented by one additional fluent
        \item Uniform axiomatization with the rest of sitcalc
        \item Ordinary actions and knowledge-producing ones are strictly separated
        \item Easy expansion of regression as defined in [Reiter2001]
        \item Desirable properties are direct consequences of the axiomatization \\
                (e.g. knowledge persistence / memory)
    \end{itemize}
\end{frame}

% opzionale
\begin{frame}{...}
    Opzionale

    Un paio di azioni ordinarie e un paio di azioni di conoscenza di esempio, giusto per inquadrare il discorso
\end{frame}

\section{Knowledge as a fluent}

\begin{frame}{The K fluent}
    \( \textfluent{K}(s', s) \)

    Defines an accessibility relation between situations.

    % Informal definition: \( \textfluent{K}(s', s) \) is true if, given its current knowledge,
    % the situations \(s\) and \(s'\) are indistinguishable to the agent.

    Informal definition: \( \textfluent{K}(s',s) \) is true if, an agent in situation \(s\)
    could mistake the current situation for the other \(s'\), given its current knowledge.
\end{frame}

\begin{frame}{Knowledge}
    A fluent is known to be true (false) in a situation \(s\) if and only if it is true (false)
    in all situations accessible from \(s\).

    Shorthand notation: \( \textbf{Knows}(\phi, s) \defeq \forall s' \: \textfluent{K}(s',s) \rightarrow \phi(s') \)
\end{frame}
